\chapter*{Abstract}

Among all types of acquired brain injury, stroke stands out as one of the primary causes of death and disability worldwide. Currently, post-stroke rehabilitation mainly relies on physical therapy, but its outcomes can be limited and often unsatisfactory. Recently, electroceutical approaches, which use electrical stimulation to the brain, have proven promising in inducing motor recovery in animal models. These methods can deliver patterns of stimuli in either an open- or closed-loop fashion to leverage synaptic plasticity. However, current stimulation protocols often yield inconsistent results, likely due to poor consideration of the intrinsic dynamics of the target system. Alongside biological models used for studying the nervous system, modern artificial models (such as ANNs) have emerged, enabling the simulation of neural networks with specific dynamics in near real-time.
\bigskip

\textit{Objective.} This thesis aims to advance the development of a real-time hardware-based Spiking Neural Network (SNN) intended for delivering personalized and biomimetic electrical stimulation therapy in the case of an ischemic lesion. Specifically, this work encompasses the characterization of an in vivo Biological Neural Network (BNN) and the fine-tuning of the SNN, named Bi{\oe}muS, which consists of single-compartment Hodgkin-Huxley neurons with highly biomimetic synapses and noise. Both spontaneous and evoked activity of the BNN are analyzed: the former is used as the target behavior in the SNN tuning phase, while the latter helps to understand how the lesion and the subsequent traditional stimulation therapies affect the interplay between the premotor and somatosensory areas.
\bigskip

\textit{Results.} The analysis of the evoked activity at the two monitored locations (premotor and somatosensory areas) revealed a significant decrease due to the lesion. Additionally traditional open-loop stimulation therapies proved ineffective in restoring pre-lesion firing levels, in contrast to activity-dependent stimulation that leverages a closed-loop paradigm. The spontaneous activity of the premotor area of anesthetized rats, considered the biological neural network, was characterized and successfully replicated by the SNN. 
\bigskip

\textit{Significance.} The potential of these results lies in enabling a novel electroceutical therapy that combines the simplicity of an open-loop system with the personalized stimulation of a closed-loop system.
