\chapter*{Conclusions and Future Perspectives}

This thesis primarily aimed to advance the development of a novel neuromodulation technique, where the stimulation would be driven by a real-time hardware-based biomimetic SNN. Thus, the spontaneous activity of the RFA area of anesthetized rats, considered as the BNN, was analyzed to provide a target for fine-tuning the SNN, named Bi{\oe}muS. Key electrophysiological features of the in vivo neural network were successfully replicated by the SNN, such as firing and bursting rates, correlation levels, burstiness index, and inter-spike interval histogram. A total of 325 SNNs were simulated, and the one that best matched the characteristics observed in the target BNN was identified through a grading system based on a set of metrics that considered the similarity between the artificial and biological neural networks.

Regarding the design aspect of the SNN, in the future, an automatic tuning mechanism could be implemented to fine-tune its parameters, enabling a more accurate replication of the BNN behavior. Additionally, instead of utilizing an algorithm that arranges neurons in a cluster-like fashion, an approach that introduces scale-free or small-world properties to the network could be considered. All these details are essential for creating a highly personalized and biomimetic SNN that faithfully reproduces the dynamics of the targeted subject and can substitute a damaged brain region.

Furthermore, considering the future use of this platform for delivering personalized stimulation, the evoked activity in the RFA and S1 areas was also analyzed. In particular, the connectivity mapping phases were considered to compute the PSTH, enabling the understanding of how the lesion and traditional neuromodulation approaches affect the interplay between RFA and S1. It was possible to highlight a critical decrease in the evoked activity in the PoL phase, however, none of the OL techniques were able to bring the activity back to the level observed in the PreL phase. Once this technology is used for new in vivo experiments, it will be possible to assess the significance of such an open-loop technique, considering also its evoked results.